%% Generated by Sphinx.
\def\sphinxdocclass{report}
\documentclass[letterpaper,10pt,dvipdfmx]{sphinxmanual}
\ifdefined\pdfpxdimen
   \let\sphinxpxdimen\pdfpxdimen\else\newdimen\sphinxpxdimen
\fi \sphinxpxdimen=.75bp\relax
\ifdefined\pdfimageresolution
    \pdfimageresolution= \numexpr \dimexpr1in\relax/\sphinxpxdimen\relax
\fi
%% let collapsible pdf bookmarks panel have high depth per default
\PassOptionsToPackage{bookmarksdepth=5}{hyperref}

\PassOptionsToPackage{booktabs}{sphinx}
\PassOptionsToPackage{colorrows}{sphinx}

\PassOptionsToPackage{warn}{textcomp}


\usepackage{cmap}
\usepackage[T1]{fontenc}
\usepackage{amsmath,amssymb,amstext}




\usepackage{tgtermes}
\usepackage{tgheros}
\renewcommand{\ttdefault}{txtt}




\usepackage{sphinx}

\fvset{fontsize=auto}
\usepackage[dvipdfm]{geometry}
\usepackage{physics}

% Include hyperref last.
\usepackage{hyperref}
% Fix anchor placement for figures with captions.
\usepackage{hypcap}% it must be loaded after hyperref.
% Set up styles of URL: it should be placed after hyperref.
\urlstyle{same}
\usepackage{pxjahyper}

\renewcommand{\contentsname}{Contents:}

\usepackage{sphinxmessages}
\setcounter{tocdepth}{1}



\title{Sphinx\sphinxhyphen{}Test}
\date{2024年02月02日}
\release{}
\author{Lomega}
\newcommand{\sphinxlogo}{\vbox{}}
\renewcommand{\releasename}{}
\makeindex
\begin{document}

\pagestyle{empty}
\sphinxmaketitle
\pagestyle{plain}
\sphinxtableofcontents
\pagestyle{normal}
\phantomsection\label{\detokenize{index::doc}}


\sphinxstepscope

\sphinxAtStartPar
これはchpa1.rstの内容です。


\chapter{h1相当の見出し}
\label{\detokenize{chap1:h1}}\label{\detokenize{chap1::doc}}
\sphinxAtStartPar
うおうお


\section{h2相当の見出し}
\label{\detokenize{chap1:h2}}
\sphinxAtStartPar
ぐえ


\subsection{h3相当の見出し}
\label{\detokenize{chap1:h3}}
\sphinxAtStartPar
ぐおん


\subsubsection{h4相当の見出しの文字列}
\label{\detokenize{chap1:h4}}
\sphinxAtStartPar
うがー。


\paragraph{h5相当の見出し}
\label{\detokenize{chap1:h5}}
\sphinxAtStartPar
むーん。


\subparagraph{h6相当の見出し}
\label{\detokenize{chap1:h6}}
\sphinxAtStartPar
こまった。

\sphinxstepscope


\chapter{Chapter 2}
\label{\detokenize{chap2:chapter-2}}\label{\detokenize{chap2::doc}}
\sphinxAtStartPar
これはchpa2.rstの内容です。

\sphinxstepscope


\section{Chapter 2\sphinxhyphen{}1}
\label{\detokenize{chapter2/chap2-1:chapter-2-1}}\label{\detokenize{chapter2/chap2-1::doc}}
\sphinxAtStartPar
これはchpa2\sphinxhyphen{}1.rstの内容です。ぐえ


\subsection{Sphinx とは}
\label{\detokenize{chapter2/chap2-1:sphinx}}
\sphinxAtStartPar
Sphinx はブラウザーで表示するドキュメント(オンラインマニュアル 等)を作成するツールです。
具体的には reStructredText を使用してマークダウン記法で書いたファイルを HTML 形式や PDF 形式などのファイルに変換するツールです。
元々は Python のドキュメント用に作成されたものですが、現在は多くのドキュメントを作成するのに使用されています。
名称は Sphinx ですが、「ホルスの目」のアイコンが使用されています。

\begin{sphinxVerbatim}[commandchars=\\\{\}]
include(\PYGZdq{}trgutils.jl\PYGZdq{})

module Potts

function weight(β; q)
  Main.Diagonal([exp(β) \PYGZhy{} 1 for \PYGZus{} in 1:q]) + ones(q, q)
end

function bulk(β; q)
  Main.bulk(weight(β; q))
end

function horizontalboundary(β; q)
  Main.horizontalboundary(weight(β; q))
end

function criticaltemperature(;q)
  1 / log(1 + √q)
end

end
\end{sphinxVerbatim}

\sphinxstepscope


\section{Chapter 2\sphinxhyphen{}2}
\label{\detokenize{chapter2/chap2-2:chapter-2-2}}\label{\detokenize{chapter2/chap2-2::doc}}
\sphinxAtStartPar
これはchpa2\sphinxhyphen{}2.rstの内容です。

\sphinxstepscope


\subsection{Miniconda}
\label{\detokenize{chapter2/uo:miniconda}}\label{\detokenize{chapter2/uo::doc}}

\subsubsection{インストール}
\label{\detokenize{chapter2/uo:id1}}
\sphinxAtStartPar
Miniconda のインストール手順を説明します。


\subsubsection{設定}
\label{\detokenize{chapter2/uo:id2}}
\sphinxAtStartPar
Miniconda の設定手順を説明します。

\sphinxstepscope


\chapter{はじめに}
\label{\detokenize{chap3:id1}}\label{\detokenize{chap3::doc}}

\section{ごあいさつ}
\label{\detokenize{chap3:id2}}
\sphinxAtStartPar
Sphinx を使用し始めてからすこし時間が経過したので、その振り返りとして本書を作成しました。
本書は Windows 環境上に「 Sphinx を導入 \(\rightarrow\) ドキュメントを作成 \(\rightarrow\) GitHub Pages 」で公開するまでの手順を説明します。
これから Sphinx を使用したいと考えている方のお力にに少しでもなれば幸いです。


\section{使用バージョン}
\label{\detokenize{chap3:id3}}
\sphinxAtStartPar
動作検証などで使用したバージョンは Sphinx 4.4 です。
Sphinx は適宜アップデートして動作検証するため、本書の最初と最後でバージョンが異なる可能性があります。


\section{免責事項}
\label{\detokenize{chap3:id4}}
\sphinxAtStartPar
本書の掲載内容はできる限り正確に保つように努めていますが、掲載内容の正確性・完全性・信頼性・最新性を保証するものではございません。
本書に起因して読者および第三者に損害が発生したとしても、筆者は責任を負わないものとします。

\sphinxstepscope


\chapter{文字の表現}
\label{\detokenize{chap4:id1}}\label{\detokenize{chap4::doc}}
\sphinxAtStartPar
1行目の文章です。
2行目の文章です。

\sphinxAtStartPar
丸竹夷二 押御池 姉三六角 蛸錦 四綾仏高 松万五条 雪駄ちゃらちゃら 魚の棚 六条 三哲 通りすぎ 七 越えれば 八九条 条東寺で とどめさす

\sphinxAtStartPar
\sphinxstyleemphasis{斜体}
\begin{itemize}
\item {} 
\sphinxAtStartPar
{\hyperref[\detokenize{chap4:onigiri}]{\sphinxcrossref{\DUrole{std,std-ref}{ロロノア・ゾロ}}}}

\end{itemize}

\sphinxAtStartPar
\sphinxcode{\sphinxupquote{コードサンプル}}
\begin{quote}\begin{description}
\sphinxlineitem{フィールドリスト1}
\sphinxAtStartPar
説明文その1

\sphinxlineitem{フィールドリスト2}
\sphinxAtStartPar
説明文その2

\sphinxlineitem{フィールドリスト3}
\sphinxAtStartPar
説明文その3

\end{description}\end{quote}

\begin{DUlineblock}{0em}
\item[] このように先頭に "|" を書くと
\item[] ラインブロックになり、
\item[] 改行を含めて、書いたとおりに表示します。
\end{DUlineblock}

\sphinxAtStartPar
ここは通常の文章です。次の行はリテラルコードブロックです。

\begin{sphinxVerbatim}[commandchars=\\\{\}]
ここからリテラルコードブロックです。
リテラルコードブロック部分の改行は、
ソースコードの内容がそのまま反映されます。
入力した文字はそのまま表示されます。箇条書きをしようとしても

\PYGZhy{} あああ
\PYGZhy{} いいい

のように、書いたとおりに表示します。
ここでリテラルコードブロックは終了です。
\end{sphinxVerbatim}

\sphinxAtStartPar
ここから通常の文章です。
\begin{itemize}
\item {} 
\sphinxAtStartPar
リストの 1 つ目です。

\item {} 
\sphinxAtStartPar
リストは先頭に "・" がつきます。

\item {} 
\sphinxAtStartPar
リストでも
改行は Sphinx 任せです。

\end{itemize}
\begin{enumerate}
\sphinxsetlistlabels{\arabic}{enumi}{enumii}{}{.}%
\item {} 
\sphinxAtStartPar
富士

\item {} 
\sphinxAtStartPar
鷹

\item {} 
\sphinxAtStartPar
なすび

\end{enumerate}
\begin{itemize}
\item {} 
\sphinxAtStartPar
親リストの 1 つめ
\begin{itemize}
\item {} 
\sphinxAtStartPar
子リストの 1 つめ

\item {} 
\sphinxAtStartPar
子リストの 2 つめ

\item {} 
\sphinxAtStartPar
子リストの 3 つめ

\end{itemize}

\item {} 
\sphinxAtStartPar
親リストの 2 つめ

\end{itemize}
\begin{enumerate}
\sphinxsetlistlabels{\arabic}{enumi}{enumii}{}{.}%
\item {} 
\sphinxAtStartPar
番号付き親リストの 1 つめ
\begin{enumerate}
\sphinxsetlistlabels{\arabic}{enumii}{enumiii}{}{.}%
\item {} 
\sphinxAtStartPar
番号付き子リストの 1 つめ

\item {} 
\sphinxAtStartPar
番号付き子リストの 2 つめ

\item {} 
\sphinxAtStartPar
番号付き子リストの 3 つめ

\end{enumerate}

\item {} 
\sphinxAtStartPar
番号付き親リストの 2 つめ

\end{enumerate}
\begin{enumerate}
\sphinxsetlistlabels{\alph}{enumi}{enumii}{}{.}%
\item {} 
\sphinxAtStartPar
番号の代わりに英小文字を使用したリストの 1 つ目です。

\item {} 
\sphinxAtStartPar
リストの番号は a. b. c. ・・・ になります。

\item {} 
\sphinxAtStartPar
改行は
やっぱり Sphinx 任せです。

\end{enumerate}
\begin{enumerate}
\sphinxsetlistlabels{\Alph}{enumi}{enumii}{}{.}%
\item {} 
\sphinxAtStartPar
番号の代わりに英大文字を使用したリストの 1 つ目です。

\item {} 
\sphinxAtStartPar
リストの番号は A. B. C. ・・・ になります。

\item {} 
\sphinxAtStartPar
改行は
やっぱり Sphinx 任せです。

\end{enumerate}

\sphinxAtStartPar
〇〇〇 を入力後 \sphinxguilabel{OK} をクリックします。
\begin{itemize}
\item {} 
\sphinxAtStartPar
\sphinxguilabel{list} ボタン :登録されている内容を一覧表示します。

\item {} 
\sphinxAtStartPar
\sphinxguilabel{search} ボタン :入力した文字列をキーワードにしてドキュメント内を検索します。

\end{itemize}

\sphinxAtStartPar
\sphinxstyleliteralstrong{\sphinxupquote{ls}} コマンド

\sphinxAtStartPar
スタックは \sphinxstyleabbreviation{LIFO} (last\sphinxhyphen{}in, first\sphinxhyphen{}out) 構造です。

\sphinxAtStartPar
nginx のメインの設定ファイルは \sphinxcode{\sphinxupquote{/etc/nginx/nginx.conf}} です。

\sphinxAtStartPar
サンプルのテキストファイルをダウンロードするには \sphinxcode{\sphinxupquote{ここをクリック}} します。

\sphinxAtStartPar
新しくテキストファイルを作成するには" \sphinxmenuselection{ファイル(F) \(\rightarrow\) 新規作成(N)} "の順に操作します。

\sphinxAtStartPar
処理を中断するときは \sphinxkeyboard{\sphinxupquote{Esc}} を押します。

\sphinxAtStartPar
二次方程式の一般形は 「 \(ax^2 + bx + c = 0\) 」 です。

\sphinxAtStartPar
Gauss積分は 「 \(\displaystyle\int_{-\infty}^\infty dx\,\mathrm{e}^{-ax^2} = \sqrt{\dfrac{\pi}{a}}\) 」 です。


\section{鬼斬りを表示}
\label{\detokenize{chap4:onigiri}}\label{\detokenize{chap4:id2}}
\noindent{\hspace*{\fill}\sphinxincludegraphics[scale=0.7]{{image}.png}\hspace*{\fill}}

\sphinxAtStartPar
Sphinx の日本ユーザー会のサイトは \sphinxurl{https://sphinx-users.jp/index.html} です。

\sphinxAtStartPar
Sphinx の日本ユーザー会のサイトは \sphinxhref{https://sphinx-users.jp/index.html}{ここをクリック} します。

\sphinxAtStartPar
このサイトについて {\hyperref[\detokenize{chap3::doc}]{\sphinxcrossref{\DUrole{doc}{はじめに}}}}


\begin{savenotes}\sphinxattablestart
\sphinxthistablewithglobalstyle
\centering
\begin{tabulary}{\linewidth}[t]{TTT}
\sphinxtoprule
\sphinxtableatstartofbodyhook
\sphinxAtStartPar
A
&
\sphinxAtStartPar
B
&
\sphinxAtStartPar
A and B
\\
\sphinxhline
\sphinxAtStartPar
False
&
\sphinxAtStartPar
うおe
&
\sphinxAtStartPar
False
\\
\sphinxhline
\sphinxAtStartPar
True
&
\sphinxAtStartPar
Fals
&
\sphinxAtStartPar
Flase
\\
\sphinxhline
\sphinxAtStartPar
False
&
\sphinxAtStartPar
True
&
\sphinxAtStartPar
False
\\
\sphinxhline
\sphinxAtStartPar
True
&
\sphinxAtStartPar
True
&
\sphinxAtStartPar
True
\\
\sphinxbottomrule
\end{tabulary}
\sphinxtableafterendhook\par
\sphinxattableend\end{savenotes}


\begin{savenotes}\sphinxattablestart
\sphinxthistablewithglobalstyle
\centering
\begin{tabulary}{\linewidth}[t]{TTT}
\sphinxtoprule
\sphinxtableatstartofbodyhook
\sphinxAtStartPar
A
&
\sphinxAtStartPar
B
&
\sphinxAtStartPar
A and B
\\
\sphinxhline
\sphinxAtStartPar
False
&
\sphinxAtStartPar
うおe
&
\sphinxAtStartPar
False
\\
\sphinxhline
\sphinxAtStartPar
True
&
\sphinxAtStartPar
False
&
\sphinxAtStartPar
False
\\
\sphinxhline
\sphinxAtStartPar
False
&
\sphinxAtStartPar
True
&
\sphinxAtStartPar
False
\\
\sphinxhline
\sphinxAtStartPar
True
&
\sphinxAtStartPar
True
&
\sphinxAtStartPar
True
\\
\sphinxbottomrule
\end{tabulary}
\sphinxtableafterendhook\par
\sphinxattableend\end{savenotes}


\begin{savenotes}\sphinxattablestart
\sphinxthistablewithglobalstyle
\centering
\begin{tabulary}{\linewidth}[t]{TTT}
\sphinxtoprule
\sphinxtableatstartofbodyhook
\sphinxAtStartPar
A
&
\sphinxAtStartPar
B
&
\sphinxAtStartPar
A and B
\\
\sphinxhline
\sphinxAtStartPar
False
&
\sphinxAtStartPar
False
&
\sphinxAtStartPar
False
\\
\sphinxhline
\sphinxAtStartPar
True
&
\sphinxAtStartPar
False
&
\sphinxAtStartPar
False
\\
\sphinxhline
\sphinxAtStartPar
False
&
\sphinxAtStartPar
True
&
\sphinxAtStartPar
False
\\
\sphinxhline
\sphinxAtStartPar
True
&
\sphinxAtStartPar
True
&
\sphinxAtStartPar
True
\\
\sphinxbottomrule
\end{tabulary}
\sphinxtableafterendhook\par
\sphinxattableend\end{savenotes}


\begin{savenotes}\sphinxattablestart
\sphinxthistablewithglobalstyle
\centering
\sphinxcapstartof{table}
\sphinxthecaptionisattop
\sphinxcaption{各年代の伝説的アニメ}\label{\detokenize{chap4:id4}}
\sphinxaftertopcaption
\begin{tabular}[t]{\X{1}{3}\X{1}{3}\X{1}{3}}
\sphinxtoprule
\sphinxstyletheadfamily 
\sphinxAtStartPar
2011年
&\sphinxstyletheadfamily 
\sphinxAtStartPar
2012年
&\sphinxstyletheadfamily 
\sphinxAtStartPar
2015年
\\
\sphinxmidrule
\sphinxtableatstartofbodyhook
\sphinxAtStartPar
まどマギ
&
\sphinxAtStartPar
中二恋
&
\sphinxAtStartPar
Charlotte
\\
\sphinxbottomrule
\end{tabular}
\sphinxtableafterendhook\par
\sphinxattableend\end{savenotes}


\begin{savenotes}\sphinxattablestart
\sphinxthistablewithglobalstyle
\centering
\begin{tabulary}{\linewidth}[t]{TTTT}
\sphinxtoprule
\sphinxtableatstartofbodyhook&
\sphinxAtStartPar
A
&
\sphinxAtStartPar
B
&
\sphinxAtStartPar
Result
\\
\sphinxhline\sphinxmultirow{4}{5}{%
\begin{varwidth}[t]{\sphinxcolwidth{1}{4}}
\sphinxAtStartPar
and
\par
\vskip-\baselineskip\vbox{\hbox{\strut}}\end{varwidth}%
}%
&
\sphinxAtStartPar
False
&
\sphinxAtStartPar
False
&\sphinxmultirow{3}{8}{%
\begin{varwidth}[t]{\sphinxcolwidth{1}{4}}
\sphinxAtStartPar
False
\par
\vskip-\baselineskip\vbox{\hbox{\strut}}\end{varwidth}%
}%
\\
\sphinxcline{2-3}\sphinxfixclines{4}\sphinxtablestrut{5}&
\sphinxAtStartPar
True
&
\sphinxAtStartPar
False
&\sphinxtablestrut{8}\\
\sphinxcline{2-3}\sphinxfixclines{4}\sphinxtablestrut{5}&
\sphinxAtStartPar
False
&
\sphinxAtStartPar
True
&\sphinxtablestrut{8}\\
\sphinxcline{2-4}\sphinxfixclines{4}\sphinxtablestrut{5}&
\sphinxAtStartPar
True
&
\sphinxAtStartPar
True
&
\sphinxAtStartPar
True
\\
\sphinxhline\sphinxmultirow{4}{16}{%
\begin{varwidth}[t]{\sphinxcolwidth{1}{4}}
\sphinxAtStartPar
or
\par
\vskip-\baselineskip\vbox{\hbox{\strut}}\end{varwidth}%
}%
&
\sphinxAtStartPar
False
&
\sphinxAtStartPar
False
&
\sphinxAtStartPar
False
\\
\sphinxcline{2-4}\sphinxfixclines{4}\sphinxtablestrut{16}&
\sphinxAtStartPar
True
&
\sphinxAtStartPar
False
&\sphinxmultirow{3}{22}{%
\begin{varwidth}[t]{\sphinxcolwidth{1}{4}}
\sphinxAtStartPar
True
\par
\vskip-\baselineskip\vbox{\hbox{\strut}}\end{varwidth}%
}%
\\
\sphinxcline{2-3}\sphinxfixclines{4}\sphinxtablestrut{16}&
\sphinxAtStartPar
False
&
\sphinxAtStartPar
True
&\sphinxtablestrut{22}\\
\sphinxcline{2-3}\sphinxfixclines{4}\sphinxtablestrut{16}&
\sphinxAtStartPar
True
&
\sphinxAtStartPar
True
&\sphinxtablestrut{22}\\
\sphinxbottomrule
\end{tabulary}
\sphinxtableafterendhook\par
\sphinxattableend\end{savenotes}


\begin{savenotes}\sphinxattablestart
\sphinxthistablewithglobalstyle
\centering
\begin{tabular}[t]{\X{1}{4}\X{1}{4}\X{2}{4}}
\sphinxtoprule
\sphinxtableatstartofbodyhook
\sphinxAtStartPar
A
&
\sphinxAtStartPar
B
&
\sphinxAtStartPar
A and B
\\
\sphinxhline
\sphinxAtStartPar
False
&
\sphinxAtStartPar
False
&
\sphinxAtStartPar
False
\\
\sphinxhline
\sphinxAtStartPar
True
&
\sphinxAtStartPar
False
&
\sphinxAtStartPar
False
\\
\sphinxhline
\sphinxAtStartPar
False
&
\sphinxAtStartPar
True
&
\sphinxAtStartPar
False
\\
\sphinxhline
\sphinxAtStartPar
True
&
\sphinxAtStartPar
True
&
\sphinxAtStartPar
True
\\
\sphinxbottomrule
\end{tabular}
\sphinxtableafterendhook\par
\sphinxattableend\end{savenotes}


\begin{savenotes}\sphinxattablestart
\sphinxthistablewithglobalstyle
\centering
\begin{tabulary}{\linewidth}[t]{TTT}
\sphinxtoprule
\sphinxstyletheadfamily 
\sphinxAtStartPar
A
&\sphinxstyletheadfamily 
\sphinxAtStartPar
B
&\sphinxstyletheadfamily 
\sphinxAtStartPar
A and B
\\
\sphinxmidrule
\sphinxtableatstartofbodyhook
\sphinxAtStartPar
False
&
\sphinxAtStartPar
False
&
\sphinxAtStartPar
False
\\
\sphinxhline
\sphinxAtStartPar
True
&
\sphinxAtStartPar
False
&
\sphinxAtStartPar
Flase
\\
\sphinxhline
\sphinxAtStartPar
False
&
\sphinxAtStartPar
True
&
\sphinxAtStartPar
False
\\
\sphinxhline
\sphinxAtStartPar
True
&
\sphinxAtStartPar
True
&
\sphinxAtStartPar
True
\\
\sphinxbottomrule
\end{tabulary}
\sphinxtableafterendhook\par
\sphinxattableend\end{savenotes}

\begin{sphinxVerbatim}[commandchars=\\\{\}]
\PYG{n}{public} \PYG{k}{class} \PYG{n+nc}{HelloWorld}\PYG{p}{\PYGZob{}}
   \PYG{n}{public} \PYG{n}{static} \PYG{n}{void} \PYG{n}{main}\PYG{p}{(}\PYG{n}{String}\PYG{p}{[}\PYG{p}{]} \PYG{n}{args}\PYG{p}{)}\PYG{p}{\PYGZob{}}
     \PYG{n}{System}\PYG{o}{.}\PYG{n}{out}\PYG{o}{.}\PYG{n}{println}\PYG{p}{(}\PYG{l+s+s2}{\PYGZdq{}}\PYG{l+s+s2}{hello, world}\PYG{l+s+s2}{\PYGZdq{}}\PYG{p}{)}\PYG{p}{;}
   \PYG{p}{\PYGZcb{}}
\PYG{p}{\PYGZcb{}}
\end{sphinxVerbatim}

\begin{sphinxVerbatim}[commandchars=\\\{\},numbers=left,firstnumber=1,stepnumber=1]
\PYG{+w}{ }\PYG{k+kd}{public}\PYG{+w}{ }\PYG{k+kd}{class} \PYG{n+nc}{HelloWorld}\PYG{p}{\PYGZob{}}
\PYG{+w}{    }\PYG{k+kd}{public}\PYG{+w}{ }\PYG{k+kd}{static}\PYG{+w}{ }\PYG{k+kt}{void}\PYG{+w}{ }\PYG{n+nf}{main}\PYG{p}{(}\PYG{n}{String}\PYG{o}{[}\PYG{o}{]}\PYG{+w}{ }\PYG{n}{args}\PYG{p}{)}\PYG{p}{\PYGZob{}}
\PYG{+w}{      }\PYG{n}{System}\PYG{p}{.}\PYG{n+na}{out}\PYG{p}{.}\PYG{n+na}{println}\PYG{p}{(}\PYG{l+s}{\PYGZdq{}}\PYG{l+s}{hello, world}\PYG{l+s}{\PYGZdq{}}\PYG{p}{)}\PYG{p}{;}
\PYG{+w}{    }\PYG{p}{\PYGZcb{}}
\PYG{+w}{ }\PYG{p}{\PYGZcb{}}
\end{sphinxVerbatim}

\fvset{hllines={, 3, 5,}}%
\begin{sphinxVerbatim}[commandchars=\\\{\}]
\PYG{n}{include}\PYG{p}{(}\PYG{l+s}{\PYGZdq{}}\PYG{l+s}{trgutils.jl}\PYG{l+s}{\PYGZdq{}}\PYG{p}{)}

\PYG{k}{module}\PYG{+w}{ }\PYG{n}{Potts}

\PYG{k}{function}\PYG{+w}{ }\PYG{n}{weight}\PYG{p}{(}\PYG{n}{β}\PYG{p}{;}\PYG{+w}{ }\PYG{n}{q}\PYG{p}{)}
\PYG{+w}{  }\PYG{n}{Main}\PYG{o}{.}\PYG{n}{Diagonal}\PYG{p}{(}\PYG{p}{[}\PYG{n}{exp}\PYG{p}{(}\PYG{n}{β}\PYG{p}{)}\PYG{+w}{ }\PYG{o}{\PYGZhy{}}\PYG{+w}{ }\PYG{l+m+mi}{1}\PYG{+w}{ }\PYG{k}{for}\PYG{+w}{ }\PYG{n}{\PYGZus{}}\PYG{+w}{ }\PYG{k}{in}\PYG{+w}{ }\PYG{l+m+mi}{1}\PYG{o}{:}\PYG{n}{q}\PYG{p}{]}\PYG{p}{)}\PYG{+w}{ }\PYG{o}{+}\PYG{+w}{ }\PYG{n}{ones}\PYG{p}{(}\PYG{n}{q}\PYG{p}{,}\PYG{+w}{ }\PYG{n}{q}\PYG{p}{)}
\PYG{k}{end}

\PYG{k}{function}\PYG{+w}{ }\PYG{n}{bulk}\PYG{p}{(}\PYG{n}{β}\PYG{p}{;}\PYG{+w}{ }\PYG{n}{q}\PYG{p}{)}
\PYG{+w}{  }\PYG{n}{Main}\PYG{o}{.}\PYG{n}{bulk}\PYG{p}{(}\PYG{n}{weight}\PYG{p}{(}\PYG{n}{β}\PYG{p}{;}\PYG{+w}{ }\PYG{n}{q}\PYG{p}{)}\PYG{p}{)}
\PYG{k}{end}

\PYG{k}{function}\PYG{+w}{ }\PYG{n}{horizontalboundary}\PYG{p}{(}\PYG{n}{β}\PYG{p}{;}\PYG{+w}{ }\PYG{n}{q}\PYG{p}{)}
\PYG{+w}{  }\PYG{n}{Main}\PYG{o}{.}\PYG{n}{horizontalboundary}\PYG{p}{(}\PYG{n}{weight}\PYG{p}{(}\PYG{n}{β}\PYG{p}{;}\PYG{+w}{ }\PYG{n}{q}\PYG{p}{)}\PYG{p}{)}
\PYG{k}{end}

\PYG{k}{function}\PYG{+w}{ }\PYG{n}{criticaltemperature}\PYG{p}{(}\PYG{p}{;}\PYG{n}{q}\PYG{p}{)}
\PYG{+w}{  }\PYG{l+m+mi}{1}\PYG{+w}{ }\PYG{o}{/}\PYG{+w}{ }\PYG{n}{log}\PYG{p}{(}\PYG{l+m+mi}{1}\PYG{+w}{ }\PYG{o}{+}\PYG{+w}{ }\PYG{o}{√}\PYG{n}{q}\PYG{p}{)}
\PYG{k}{end}

\PYG{k}{end}
\end{sphinxVerbatim}
\sphinxresetverbatimhllines

\sphinxAtStartPar
"c:\textbackslash{}windows" ディレクトリーです。

\sphinxAtStartPar
なんだ、かんだ


\bigskip\hrule\bigskip


\sphinxAtStartPar
あーだ、こーだ

\begin{sphinxadmonition}{attention}{注意:}
\sphinxAtStartPar
如月アテンション

\sphinxAtStartPar
attentionの中にtは3回登場します。
\end{sphinxadmonition}

\sphinxAtStartPar
\(\displaystyle\int\dd[3]{x}\)
\begin{align}
  \int_{-\infty}^\infty\dd{x}\mathrm{e}^{-ax^2} = \sqrt{\dfrac{\pi}{a}} \label{eq:uo}
\end{align}
\sphinxAtStartPar
式 \(\ref{eq:uo}\) がGauss積分。


\chapter{Indices and tables}
\label{\detokenize{index:indices-and-tables}}\begin{itemize}
\item {} 
\sphinxAtStartPar
\DUrole{xref,std,std-ref}{genindex}

\item {} 
\sphinxAtStartPar
\DUrole{xref,std,std-ref}{modindex}

\item {} 
\sphinxAtStartPar
\DUrole{xref,std,std-ref}{search}

\end{itemize}



\renewcommand{\indexname}{索引}
\printindex
\end{document}